\documentclass[SE,authoryear,toc]{lsstdoc}
% lsstdoc documentation: https://lsst-texmf.lsst.io/lsstdoc.html
\input{meta}

% Package imports go here.

% Local commands go here.

%If you want glossaries
%\input{aglossary.tex}
%\makeglossaries

\title{Model for Community Engagement in Rubin Observatory Commissioning Science Validation}

% Optional subtitle
% \setDocSubtitle{A subtitle}

\author{%
Keith Bechtol
}

\setDocRef{SITCOMTN-010}
\setDocUpstreamLocation{\url{https://github.com/lsst-sitcom/sitcomtn-010}}

\date{\vcsDate}

% Optional: name of the document's curator
% \setDocCurator{The Curator of this Document}

\setDocAbstract{%
We propose a model for collaboration between the Vera C. Rubin Observatory and the science community to enhance science validation activities during commissioning.
}

% Change history defined here.
% Order: oldest first.
% Fields: VERSION, DATE, DESCRIPTION, OWNER NAME.
% See LPM-51 for version number policy.
\setDocChangeRecord{%
  \addtohist{1}{YYYY-MM-DD}{Unreleased.}{Keith Bechtol}
}


\begin{document}

% Create the title page.
\maketitle
% Frequently for a technote we do not want a title page  uncomment this to remove the title page and changelog.
% use \mkshorttitle to remove the extra pages

% ADD CONTENT HERE
% You can also use the \input command to include several content files.

\appendix
% Include all the relevant bib files.
% https://lsst-texmf.lsst.io/lsstdoc.html#bibliographies
\section{References} \label{sec:bib}
\renewcommand{\refname}{} % Suppress default Bibliography section
\bibliography{local,lsst,lsst-dm,refs_ads,refs,books}

% Make sure lsst-texmf/bin/generateAcronyms.py is in your path
\section{Acronyms} \label{sec:acronyms}
\addtocounter{table}{-1}
\begin{longtable}{p{0.145\textwidth}p{0.8\textwidth}}\hline
\textbf{Acronym} & \textbf{Description}  \\\hline

CBP & Collimated Beam Projector \\\hline
ComCam & The commissioning camera is a single-raft, 9-CCD camera that will be installed in LSST during commissioning, before the final camera is ready. \\\hline
DP0 & Data Preview 0 \\\hline
DP1 & Data Preview 1 \\\hline
DP2 & Data Preview 2 \\\hline
DR1 & Data Release 1 \\\hline
FTE & Full-Time Equivalent \\\hline
LPM & LSST Project Management (Document Handle) \\\hline
LSST & Legacy Survey of Space and Time (formerly Large Synoptic Survey Telescope) \\\hline
NIR & Near Infrared \\\hline
OSI & open systems interconnect \\\hline
PDF & Portable Document Format \\\hline
PSF & Point Spread Function \\\hline
QA & Quality Assurance \\\hline
RDO & Rubin Directors Office \\\hline
SCOC & Survey Cadence Optimization Committee \\\hline
SE & System Engineering \\\hline
SIT & System Integration, Test \\\hline
SLAC & SLAC National Accelerator Laboratory \\\hline
US & United States \\\hline
\end{longtable}

% If you want glossary uncomment below -- comment out the two lines above
%\printglossaries





\end{document}
