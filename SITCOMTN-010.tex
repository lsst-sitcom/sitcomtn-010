\documentclass[SE,authoryear,toc]{lsstdoc}
% lsstdoc documentation: https://lsst-texmf.lsst.io/lsstdoc.html
\input{meta}

% Package imports go here.

% Local commands go here.

%If you want glossaries
%\input{aglossary.tex}
%\makeglossaries

\title{Model for Community Engagement in Rubin Observatory Commissioning Science Validation}

% Optional subtitle
% \setDocSubtitle{A subtitle}

\author{%
Keith Bechtol
}

\setDocRef{SITCOMTN-010}
\setDocUpstreamLocation{\url{https://github.com/lsst-sitcom/sitcomtn-010}}

\date{\vcsDate}

% Optional: name of the document's curator
% \setDocCurator{The Curator of this Document}

\setDocAbstract{%
We propose a model for collaboration between the Vera C. Rubin Observatory and the science community to enhance science validation activities during commissioning.
}

% Change history defined here.
% Order: oldest first.
% Fields: VERSION, DATE, DESCRIPTION, OWNER NAME.
% See LPM-51 for version number policy.
\setDocChangeRecord{%
  \addtohist{1}{YYYY-MM-DD}{Unreleased.}{Keith Bechtol}
}


\begin{document}

% Create the title page.
\maketitle
% Frequently for a technote we do not want a title page  uncomment this to remove the title page and changelog.
% use \mkshorttitle to remove the extra pages

% ADD CONTENT HERE
% You can also use the \input command to include several content files.

\section{Scope}

This document proposes an integrated approach to commissioning science validation for the Vera C. Rubin Observatory that leverages scientific community involvement and expertise.
%a model for collaboration between the Vera C. Rubin Observatory and the science community to enhance science validation activities during commissioning. 
The overall plan for System Integration and Commissioning is described in LSE-79, and the overall plan for Commissioning Science Validation is described in LSE-439. 

\section{Motivation}

The Vera C. Rubin Observatory Construction Project has planned a series of in-dome calibration activities and on-sky observing campaigns during the system integration and commissioning period that are designed to explore the range of demonstrated technical and scientific performance of the as-built system (LSE-79). Three topics to be addressed during the commissioning period include system verification, validation, and characterization.

\begin{itemize}

\item \textbf{Verification:} Does the system meet the requirement specifications defined in the LSST System Requirements (LSE-29) and Observatory System Specification (LSE-30) documents?

\item \textbf{Validation:}  Does the system do what we want and/or expect it to do? Do the delivered data products and data access services meet the needs of the science community?

\item \textbf{Characterization:} Do we understand why the systems behaves the way that is does? How can the system be optimized to enhance science performance and observing efficiency?

\end{itemize}

As both stakeholders and science domain experts, members of the broad science community are well positioned to engage in science validation activities. 

%We describe an integrated approach to commissioning that also leverages community involvement and expertise.
%The larger scientific community, 
%Members of the broad science community are core stakeholders in the Rubin Observatory and have expertise 

Some specific motivations to :

\begin{itemize}
\item to inform on-sky observation strategy during commissioning (e.g., optimal dithering strategy for DDFs)
\item to contribute to science validation through code contributions / algorithm contributions
\item to contribute to science validation via data analysis (i.e., data access, use, vetting, documentation)
\item to contribute towards LSST DR1 by developing quantities such as photo-z
\item to provide practice for the Early Operations team supporting the science community
\item to accelerate early science by gaining familiarity with data model and data characteristics
\item to deliver early science based on commissioning data
\item to facilitate opportunities for Project members to be participate in science and be recognized for their contributions
\end{itemize}

A primary motivation in my mind to help make LSST DR1 as good as it can be by engaging with the community early. The intent of this model is that it is beneficial to the Project, and hopefully accelerates early scientific utilization of Rubin data.

\section{Levels of Engagement}

We anticipate at least three levels of science community engagement with Rubin Observatory commissioning data. 

\begin{enumerate}

\item Non-staff members that collaborate directly with the Project Commissioning Team, performing work directed by the Project with specific deliverables and timelines. 
Examples include International Contributions for Data Rights and other groups that establish MoUs with the Project (exact process TBD).

\item Friendly science users who access and analyze commissioning data and actively participate in joint science validation activities with the Commissioning Team (described below). 
%There is no formal expectation of support, 
The level of support for early data access prior to the Data Previews is on a best-effort basis.
%, and and collaboration is for mutual benefit. 
The data exploration by science community members is not directed by the Project, although some coordination and collaboration is likely to be generally beneficial.
The Project will likely need to prioritize what feedback is acted upon.

\item General science users of Data Previews that perform science analysis, but do not necessarily directly engage with the Commissioning Team. 
This more indirect interaction is still beneficial to the Project as a preview of realistic science workflows and results.

\end{enumerate}

A main goal of the 

\section{Data Access}

\subsection{Data Previews}

The Early Operations team has planned three more structured data releases (reference for Data Previews):

\begin{itemize}
\item Data Preview 0: based on DESC DC2 image simulations
\item Data Preview 1: based on ComCam
\item Data Preview 2: based on LSSTCam
\end{itemize}

Data Previews 1 and 2 are planned to appear roughly 6 months after the completion of the associated observing campaign.

\subsection{Rapid Data Previews}

In addition to the three planned Data Previews, we introduce the concept of a series of incremental data releases made available to the science community on a faster timescale, though in a less supported fashion. 
The Rapid Data Previews could be as small as a few ComCam visits, and as large as a candidate release for Data Preview 2. 
The data products are expected to be the same as those used by the Project Commissioning, and will be provided to the science community ``as is'' to facilitate rapid exploration.
%smaller datasets that can be acquired, reduced, vetted, and made available to the science community on a much faster timescale in a less supported fashion.


\section{Project and Science Community Interactions}



\section{Anticipated Publications}

\appendix
% Include all the relevant bib files.
% https://lsst-texmf.lsst.io/lsstdoc.html#bibliographies
\section{References} \label{sec:bib}
\renewcommand{\refname}{} % Suppress default Bibliography section
\bibliography{local,lsst,lsst-dm,refs_ads,refs,books}

% Make sure lsst-texmf/bin/generateAcronyms.py is in your path
\section{Acronyms} \label{sec:acronyms}
\addtocounter{table}{-1}
\begin{longtable}{p{0.145\textwidth}p{0.8\textwidth}}\hline
\textbf{Acronym} & \textbf{Description}  \\\hline

CBP & Collimated Beam Projector \\\hline
ComCam & The commissioning camera is a single-raft, 9-CCD camera that will be installed in LSST during commissioning, before the final camera is ready. \\\hline
DP0 & Data Preview 0 \\\hline
DP1 & Data Preview 1 \\\hline
DP2 & Data Preview 2 \\\hline
DR1 & Data Release 1 \\\hline
FTE & Full-Time Equivalent \\\hline
LPM & LSST Project Management (Document Handle) \\\hline
LSST & Legacy Survey of Space and Time (formerly Large Synoptic Survey Telescope) \\\hline
NIR & Near Infrared \\\hline
OSI & open systems interconnect \\\hline
PDF & Portable Document Format \\\hline
PSF & Point Spread Function \\\hline
QA & Quality Assurance \\\hline
RDO & Rubin Directors Office \\\hline
SCOC & Survey Cadence Optimization Committee \\\hline
SE & System Engineering \\\hline
SIT & System Integration, Test \\\hline
SLAC & SLAC National Accelerator Laboratory \\\hline
US & United States \\\hline
\end{longtable}

% If you want glossary uncomment below -- comment out the two lines above
%\printglossaries





\end{document}
